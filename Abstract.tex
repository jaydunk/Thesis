At sufficiently high temperatures and densities quarks and gluons exist in a deconfined state called Quark Gluon Plasma (QGP). QGP existed in the Universe shortly after the Big Bang, and today is created in accelerator based experiments which collide heavy nuclei at high energies. Results from these experiments point to a hot, dense and strongly interacting state of deconfined quarks and gluons. The study of heavy flavor probes (those originating from $c$ and $b$ quarks) is an active area of research in heavy ion collisions. Heavy quarks are produced in the initial hard scatterings of collisions and thus are sensitive to the entire evolution of the medium. They also potentially have different sensitivity to medium induced energy loss compared to light flavors. 

This dissertation investigates the interactions of heavy flavor quarks with the medium by studying correlations between electrons from heavy flavor decays and hadrons. At high transverse momentum, the direction of the electron is highly correlated with the direction of the parent heavy flavor meson. We look for evidence of energy loss in the QGP as well as jet induced effects on the medium. We present electron-hadron correlations from Au+Au collisions in a wide range of centrality bins as well as correlations from p+p. The datasets used are the best currently available due to high statistics and low material in the detector. We also investigate the dependence on the orientation of the trigger particle to the event plane to look for path length dependent effects on the correlation as well as non-flow contributions to electron electron $v_2$.

