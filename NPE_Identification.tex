\chapter{Identification of Non-photonic Electrons}

We discuss the procedure for identifying electrons in events at STAR and how we remove photonic background. We show the event and track selection criteria and then lastly we will analyze the efficiency for identifying background photonic electrons. The identification of non-photonic electrons (NPE) and efficiency thereof will be critical factors when we construct the NPE-hadron correlations in later chapters.

\section{Outline of the NPE Identification}

This chapter will lay out the general methods for event selection, track selection, electron identification, and the removal of photonic electron background for both Au+Au and p+p collisions.

We start by identifying the dataset and the trigger collections we will use for the analysis. We look at the events and check that the quality of the event is good and that there could be candidate tracks for NPE in the event. We then reconstruct all tracks in the TPC and apply track quality cuts. To identify electrons we rely on the energy loss ($dE/dx$) measured in the TPC and on the hits in the EMC towers and shower max detector. 

The background from photonic electrons will be removed by searching for the opposite signed partner electron. If the primary track is from Dalitz decays or photon conversion in the detector, the partner and primary track should have a low invariant mass. We will also investigate, through simulations, the efficiency for determining the background from photonic electrons.

In the end we will have a sample of electrons which we can use as triggers for measuring NPE-hadron correlations. 

\section{Dataset and Event Selection}

\section{Track Reconstruction and TPC Cuts}

\section{BEMC Points and Matching}

\section{Electron Purity}

\section{Photonic Electron Identification}

\section{Photonic Electron Reconstruction Efficiency}
