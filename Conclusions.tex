\chapter{Conclusion}

We have made measurements of azimuthal correlations between non-photonic electrons and hadrons in both Au+Au and p+p collisions. Results have improved statistics over previous studies allowing for more us to examine the correlations across centrality and $p_T$ bins. This was possible due to the large statistics collected in STAR run 11 as well as improved condition of the TPC relative to run 10. We compared the away side jet shapes in Au+Au to collisions in p+p. We see some evidence for broadening of the away side peak in central Au+Au collisions. We also looked at the ratio of yields ($I_{AA}$) between Au+Au and p+p as a function of associated particle $p_T$ and collision centrality, but did not see any strong trend in these measurements. 

We have also looked at correlations relative to the event plane to look for direction dependent jet suppression as a result of different path lengths through the medium. Interpretation of the results is difficult in part due to large systematic errors mainly from uncertainty in the non-photonic electron $v_2$, but in the one bin with large statistics and small systematic error we see no significant difference between the in-plane and out-of-plane cases unlike what was seen in dihadron correlations. The possibility still remains that the correlation shapes are the result of the decay products of the parent mesons. The relation of the kinematics of the final state particles to the underlying hard process are also different between light flavor and heavy flavor and this further complicates direct comparisons between the two measurements.

Upgrades at STAR and RHIC should improve our ability to measure two particle correlations in the heavy flavor sector and improve the insights we can make. The current STAR Au+Au program will provide a dataset even larger than run 11 and with the Heavy Flavor Tracker will greatly increase the ability to reconstruct $D$ mesons. With these improved tools we can hope to construct two particle correlations ($D$-h, e-e, e-$\mu$, $D$-$D$, etc.) which more directly probe the kinematics of the initial heavy quarks and create measurements which are easier to interpret and compare with theoretical models of heavy quark energy loss in quark gluon plasma.
