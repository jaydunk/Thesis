\chapter{Azimuthal Correlations of Non-Photonic Electrons to Hadrons}

We will now investigate the correlations of triggered non-photonic electrons to hadrons in $Au+Au$ and $p+p$ collisions at 200 GeV. Hard processes in these collisions will produce back to back jets in the azimuthal angle $\phi$. We search for potential modification of the jet in $Au+Au$ collisions compared to $p+p$. 

\section{Overview of Constructing the NPE-hadron Correlation}

Several steps are needed to produce the NPE-h correlation. The trigger particle electrons are identified by the procedure described in the previous chapter. The nonuniform acceptance of detector results in false correlations which are not a result of the underlying physics. This is corrected in two ways, the $\phi$ distribution of all particles in flattened and then the correlations from mixed events are calculated and then a weighting is determined so as to flatten these as well.

In correlations from $Au+Au$ collisions there is an underlying background correlation from the flow of the both the trigger electron and the associated hadron. In this analysis we only consider the second order harmonic of flow, $v_{2}$. For hadrons, $v_{2}$ is very accurately measured across a wide range centralities and $p_{T}$. For non-photonic electrons, the measurements of $v_{2}$ are not so precise, thus we can only estimate its contribution to the background. This uncertainty will be reflected in the analysis of systematic error.

We will also look at the dependence of the correlation on the angle between the triggered electron and the event plane. A dependence on this angle could point to path length dependence on the jet suppression in QGP.

\section{Single Particle $\phi$-weighting}



\section{Mixed Event Background}

\section{Background from Flow}

\section{Correlations in Au+Au}

\section{Correlations in p+p}

\section{Event-Plane Dependent Correlations}

\section{Event Plane Reconstruction}

